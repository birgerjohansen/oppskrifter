\documentclass[11pt,]{article}
\usepackage{lmodern}
\usepackage{amssymb,amsmath}
\usepackage{ifxetex,ifluatex}
\usepackage{fixltx2e} % provides \textsubscript
\ifnum 0\ifxetex 1\fi\ifluatex 1\fi=0 % if pdftex
  \usepackage[T1]{fontenc}
  \usepackage[utf8]{inputenc}
\else % if luatex or xelatex
  \ifxetex
    \usepackage{mathspec}
    \usepackage{xltxtra,xunicode}
  \else
    \usepackage{fontspec}
  \fi
  \defaultfontfeatures{Mapping=tex-text,Scale=MatchLowercase}
  \newcommand{\euro}{€}
    \setmainfont{Hoefler Text}
\fi
% use upquote if available, for straight quotes in verbatim environments
\IfFileExists{upquote.sty}{\usepackage{upquote}}{}
% use microtype if available
\IfFileExists{microtype.sty}{%
\usepackage{microtype}
\UseMicrotypeSet[protrusion]{basicmath} % disable protrusion for tt fonts
}{}
\usepackage[margin=1in]{geometry}
\ifxetex
  \usepackage[setpagesize=false, % page size defined by xetex
              unicode=false, % unicode breaks when used with xetex
              xetex]{hyperref}
\else
  \usepackage[unicode=true]{hyperref}
\fi
\hypersetup{breaklinks=true,
            bookmarks=true,
            pdfauthor={},
            pdftitle={},
            colorlinks=true,
            citecolor=blue,
            urlcolor=blue,
            linkcolor=magenta,
            pdfborder={0 0 0}}
\urlstyle{same}  % don't use monospace font for urls
\setlength{\parindent}{0pt}
\setlength{\parskip}{6pt plus 2pt minus 1pt}
\setlength{\emergencystretch}{3em}  % prevent overfull lines
\setcounter{secnumdepth}{0}

\date{}

\begin{document}

{
\hypersetup{linkcolor=black}
\setcounter{tocdepth}{3}
\tableofcontents
}
\clearpage

\section{FISKEKAKER}\label{fiskekaker}

\emph{4 personer}

\begin{verbatim}
  500 g seifilet. Du kan også bruke torsk, hyse, laks eller uer
  1 ss salt
  1 ts malt hvit pepper
  2 ts potetmel
  2 egg
  1 dl melk
  2 dl fløte
  100 g smør
  1 løk
  2 ss gressløk
  1/2 dl solsikkeolje
\end{verbatim}

Renskjær seifileten så den blir fri for ben og skinn. Del den opp i små
terninger og ha dem i en food-processor. Det er viktig at fisken er
kald, det bedrer bindeevnen til fiskekjøttet. Kjør med salt, pepper og
potetmel til du får en fin farse. Ha i eggene og kjør dem inn. Spe med
melk og fløte til en smidig farse. Smak eventuelt til med mer salt og
pepper. Finhakk løken og fres den myk i smøret. Avkjøl løk og smør og
tilsett finhakket gressløk. Når dette er kaldt vendes det inn i
fiskefarsen. Myk løk og smør gjør kakene saftige når de er stekt. Farsen
kan se skilt ut, men fortvil ikke, dette går bra! Form kakene med en
spiseskje og klapp dem flate, ca. to cm tykke. Stek dem i olje i tre
minutter på hver side.

\subsection{Dilldressing}\label{dilldressing}

\begin{verbatim}
3 dl seterrømme
1 bunt dill
1 ss sukker
1 ss hvitvinseddik
salt og pepper
\end{verbatim}

Finhakk dill og bland den inn i rømmen sammen med sukker og eddik. Smak
til med salt og pepper.

\subsection{Grønnsaker som passer
til}\label{gruxf8nnsaker-som-passer-til}

\begin{itemize}
\itemsep1pt\parskip0pt\parsep0pt
\item
  Tomat og løksalat
\item
  Marinerte nypoteter med soltørkede tomater
\item
  Agurksalat
\end{itemize}

\emph{Tips!}

Fiskekakene kan gjerne lages i god tid og varmes opp før servering. De
kan også fryses.

\clearpage

\section{PULLED PORK}\label{pulled-pork}

\begin{verbatim}
  1 large joint of pork, such as a hand of pork

  For the marinade:
  2 tbsp honey
  2 tbsp Worcestershire sauce
  2 tbsp smoked paprika
  2 tbsp olive oil
  1 whole bulb of garlic, crushed by hand
  Plenty of fresh thyme
  Plenty of fresh rosemary
  2 big pinches of salt
  1 big pinch of pepper
  330ml bottle American pale ale
  Homemade pittas, to serve
\end{verbatim}

Put the pork into an oven tray. Start by scoring the pork along the skin
as if making crackling. Mix all the marinade ingredients together in a
bowl then cover the pork in the marinade. Rub the sauce and the garlic
into the scored slits in the meat and leave the to marinate overnight.

The next day preheat the oven to 240°C/gas mark 9. Put the pork into the
oven for 20 minutes or until it gets a great colour. After 20 minutes
take the meat out, cover it in tin foil and put it back into the oven at
a cooler 140°C/gas mark 1 for 8 hours to slow roast.

Eight hours later it should pull from the bone and be lovely and tender.
Cut the pittas open down one side and fill with pulled pork. Drizzle
some of the marinade sauce over the top and serve with coleslaw

\subsection{Pita bread}\label{pita-bread}

\begin{verbatim}
  1 tsp dried yeast
  300 ml tepid water
  560g strong white bread flour
  Pinch of salt
  2 good glugs of olive oil
\end{verbatim}

Mix the dried yeast into the warm water and set to one side. Put the
flour into a bowl, add the salt then mix in the yeasty solution. Add the
olive oil and knead it all together until it is smooth and firm. Leave
it to rise for half an hour. Once it's risen, cut the dough into 8,
shape into balls and roll out on both sides. You need your oven at the
hottest it will go -- 240°C/gas mark 9, or higher if you can. Put the
pittas onto a preheated heavy metal baking tray or a baking stone and
pit them into the hot oven until they blister up - it will be around 2
minutes. Any longer and they'll start to burn.

\end{document}
